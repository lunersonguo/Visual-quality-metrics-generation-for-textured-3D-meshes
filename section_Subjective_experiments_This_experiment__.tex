\section{Subjective experiments}
This experiment aims at studying the visual annoyance impacts from mixed textural and geometric artifacts on textured 3D models. In detail, we ask all the participants firstly to observe a reference model briefly, and then, regarding to the reference, select out the more similar model in the following paired comparisons. The outputs of this experiment show the patterns on the HVS sensitivity towards different types of artifacts, and can be also used as an evaluation on the performance of existing metrics.
\subsection{Experimental design}
In the design of subjective quality evaluation experiment, we selected the protocol, known as paired comparison \cite{Lee_2011}, over rating methods which usually have problems with validity and reliability \cite{Govindarajulu_1992}. Opposite to the rating methods, for which the subjects usually require necessary trainings before the experiment, paired comparisons have the advantages of more simplicity and subject consistency \cite{Wills_2009}. \\
In the study, each observer was firstly shown a rendered video clip containing illumination and a textured 3D mesh with constant speed rotation. After observation within15 seconds, a series of paired video clips having the same illumination and animating object, but with varying amount of (geometric and textural) distortions were displayed to the observer, and he/she was asked to select a more similar clip from left side video or right side one, regarding as the reference video. 