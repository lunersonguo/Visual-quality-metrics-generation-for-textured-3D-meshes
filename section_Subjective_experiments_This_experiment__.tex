\section{Subjective experiments}
This experiment aims at studying the visual annoyance impacts from mixed textural and geometric artifacts on textured 3D models. In detail, we ask all the participants firstly to observe a reference model briefly, and then, regarding to the reference, select out the more similar model in the following paired comparisons. The outputs of this experiment show the patterns on the HVS sensitivity towards different types of artifacts, and can be also used as an evaluation on the performance of existing metrics.
\subsection{Experimental design}
In the design of subjective quality evaluation experiment, we selected the protocol, known as paired comparison \cite{Lee_2011}, over rating methods which usually have problems with validity and reliability \cite{Govindarajulu_1992}. Opposite to the rating methods, for which the subjects usually require necessary trainings before the experiment, paired comparisons have the advantages of more simplicity and subject consistency \cite{Wills_2009}. \\
In the study, each observer was firstly shown a rendered video clip containing illumination and a textured 3D mesh with constant speed rotation. After an observation within 15 seconds, a series of paired video clips having the same illumination and animating object, but with varying amount of (geometric and textural) distortions were displayed to the observer, and he/she was asked to select a more similar clip from left side video or right side one, regarding to the reference video (Figure 1 presents a screenshot of the experiment test). \\
For the reference models, we chose five textured triangle meshes from different applied technological fields. The Hulk and the sport car are the artificial models from ShareGC.com, which is a social community of CG Artists, Animators and 3D Modelers displaying and sharing their work. The squirrel and the Easter Island statue, supported by Computer Graphics and Geometry Laboratory, are scanned real objects mapped with texture extracted from multiple photographs of physical statues. The Dwarf is also an artificial textured mesh created by. (Figure 2 shows all our models for the experiment)\\
All the models were rendered in Autodesk 3DS Max 2015, commonly used as a professional  3D computer graphics program for making 3D animations, models, games and images. We used two sets of illumination and rendering settings to render the video clips.   Skylight illumination was firstly used to cast on all the models and their distorted ones since the surface properties are clearer under this illumination than under atypical ones such as a single point light source \cite{Fleming_2003}. Then, we utilized a spotlight illumination, set up by , on Dwarf models for validating the feasibility of our combined metrics under different illuminations.  Two low-speed rotating rates by pivoting the vertical axis were applied to the models under two different illuminations respectively, 0.628rad/s and 0.838rad/s, besides participants could optionally control the video play to observe models carefully. The sizes of rendered videos under two illuminations are 1280X1024 and 1920X1080(HDTV). All the experiments were conducted on a 15-inche MacBook Pro.\\
\subsection{Stimuli creation and selection}
It is essential to choose appropriate distortions for our subjective study.  The purpose of this experiment is to capture the human perceptibility toward complex textured artifacts. Thus, we applied several kinds of distortions on mesh and texture separately and then combinatorially mapped the textures (including original texture) to the corresponding meshes (including original mesh). Each distortion degree can be applied in varying amount, and we applied the distortion degrees by the following criteria: \\
\begin{itemize}
\item For each dataset, present diversified degrees of distortions in order to conclude significant remarks after the analysis of metrics performances. 
\end{itemize}
\begin{itemize}
\item For each texture or mesh, find empirically increasing strength of distortions, which are clearly visible but not too strong, to avoid the biasing subjective results.
\end{itemize}
\begin{itemize}
\item For the experiment time, in each dataset, use limited number of distorted models to restrict the comparison time within an acceptable range by observers. 
\end{itemize}
For the criteria, we selected 5 commonly used geometric and textural distortions, solely applied each of them to each model in 4 different strengths based on our empirical decisions, and generated 20 distorted models for one reference model, In details, the original mesh or texture was firstly duplicated 20 times, and then each copy was processed by one single strength of the algorithm. We used following processing operations to generate all the distorted models:\\
for each mesh,
\begin{itemize}
\item \textbf{Compression}:  we used common lossy process of  compression algorithms to generate uniform geometric quantization.
\end{itemize}
\begin{itemize}
\item  \textbf{Simplification}: We applied quadric based mesh decimation method \cite{Garland_1997} to generate the simplified models in this study.
\end{itemize}
\begin{itemize}
\item \textbf{Smoothing}:We considered Laplacien smoothing algorithm \cite{Steinbrecher_2008}.
\end{itemize}
for each texture,
\begin{itemize}
\item \textbf{JPEG}: A commonly used algorithm of lossy compression for 2D images was used to compress each texture.
\end{itemize}
\begin{itemize}
\item \textbf{Sub-sampling}: We reduced the texture size by resampling through bilinear interpolation .
\end{itemize}
\subsection{Participants and backstage procedure}
We invited 85 participants to take part in the experiment, aging from 20 to 55 years, and they have normal or corrected vision. All the participants were the students and teaching staffs from the University of Lyon in France and University of Alberta in Canada, and 9 of them have the professional experience on image processing. On average, there are 12 minutes for each observer to finish the experiment for one reference model.  In general, it took longer time for observers to compare those models (squirrel and Easter Island statue) with almost monochrome texture than others.\\
Since the task of each observer in paired comparisons is to provide a preference to one distorted model, which is closer to the reference, the number of comparisons (“rounds”) determines the reliability of subjective quality opinion. Ideally, when \textit{N} distortions are evaluated, the total number of rounds is $C^N_2$, which means to show all the possible combinations of two distorted models from \textit{N} distortions. However when \textit{N} is large (e.g., 20 in our study), the number of pairs become too many (e.g., 190 in our study) to be feasible \cite{Lee_2011}, which means each subject may spend much time on some redundant comparisons. For instance, two models with same distortion but different strengths, the one with lower distorting strength is statistically preferred than the other. Thus, the total comparisons of all the possible combinations are lack of efficiency, and the redundant comparisons may distract subjects from the valuable ones.\\
In such cases, we used sorting method \cite{Farrell_2001} to reduce comparisons number and improve the efficiency. Detailedly, we firstly divided 20 distorted models into 5 groups by their distortion types (compression, simplification, smoothing, JPEG and Sub-sampling). In each group, we sorted the 4 distorted models by the increasing strengths of distortions (The distortion strengths are from 1 to 4, higher value means higher strength degree). According to the sorting method, the quality distances (differences) among groups are smaller than those inside each group, and thus we can conduct far fewer comparisons between models from two groups, and the quality estimation inside each group can be obtained based on the smaller distances determined by subjects. For instance, the model with distortion\textit{ smoothing 1} is preferred by the subject than the model with \textit{compression 2,} there will be no need to compare the distortions between \textit{smoothing 1} and \textit{compression 3} or \textit{compression 4,} since higher distortion strength of same type causes worse visual quality, hence the visual quality rank of these 4 distortions can be deduced and sorted as \textit{smoothing 1} > \textit{compression 2} > \textit{compression 3} > \textit{compression 4}. Figure 3, figure 4 and figure 5 show the detailed steps about the comparisons backstage procedure embedded with this sorting method. For the sake of convenience in statement, we use abbreviate letters with number, standing for different distortions with different strengths. \textit{Q} stands for quantization compression, \textit{J} for JPEG, \textit{L} for Laplacien smoothing, \textit{Si} for simplification and \textit{Su} for sub-sampling, and higher number means higher distortion strength.\\
\subsection{Analysis}
This experimental protocol significantly optimizes the comparisons for each observer, and thus improves the efficiency.  While the number of $C^N_2$ comparisons is 190, in our study each subject maximally does 44 comparisons to rank all the 20 models. In fact, throughout all the experiments, the average comparison number is 36.\\
\subsubsection{User agreements}
It is essential to analyze the agreements between all the subjects before we study the results from the experiment.  We utilized Kendall’s Coefficient concordance \textit{W} to assess the consistency of the ranks from the subjects.  \textit{W} ranges from 1, meaning complete agreement, to 0, meaning no agreement, while p-value associated with \textit{W} provides the likelihood of null hypothesis, which means no agreement between all the subjects. If p-value is less than a predetermined level of significance (usually determined as 0.001), we reject the null hypothesis. For each reference model, we computed the Kendall’s \textit{W} between the subjective ranks of 20 distorted models given by all the subjects (Table 1).


