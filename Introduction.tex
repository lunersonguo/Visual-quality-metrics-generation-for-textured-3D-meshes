\section{Inrtoduction}
Texture mapped 3D graphics data are extensively used in various technology realms, such as scientific visualization, virtual reality, video gaming, engineering design, medical care, and internet industry. For different purposes, these textured 3D models are processed by diversified operations (e.g. compression, simplification, filtering, watermarking, etc on geometry; JPEG compression, sub-sampling, etc on texture), which are not only restricted to alter the geometry but also the texture. Consequentially, these processing algorithms cause the perceptual annoyance on the models’ final rendered appearance, and, in order to boost/evaluate the processing operations, there merge various metrics to measure the distortions caused by the graphics attacks. In the research of the geometric study, researchers\cite{Lavou__2011}\cite{V_a_2012}\cite{Wang_2011} proposed many kinds of mesh visual quality (MVQ) metrics recently. Previous studies\cite{Corsini_2013}\cite{Guo_2015}show that the results from these metrics are highly correlated with the human opinion, compared to classical metrics, such as Hausdorff distance and root mean square error. However, for textured 3D models, the geometric properties may no longer have the dominant visual impacts since they can be compensated by their textures.  On the aspect of the texture, comprehensive studies on 2D images have paved the way for evaluating the graphics attacks on texture.  Similar to the 3D metrics, image visual quality metrics, such as structural similarity (SSIM) \cite{Wang_2004}, multi-scale structural similarity (MS-SSIM)\cite{Zhou_Wang_2011}, etc, produce better results than classical metrics (e.g., mean square error (MSE) and peak signal to noise ratio (PSNR), etc) in term of correlation with the human opinion. However, similarly, the problem becomes complex when the operations cause the distortions on geometry or/and texture. Thus, solely using single metric (either for the mesh or for the texture) is not reasonable. Towards the textured 3D model evaluation, some researches focused on the final displays of the models. For instance, video quality assessment metrics (MSU VQMT, etc), inherited from image quality assessment metrics\cite{Wang_2006}, evaluate the videos, containing textured 3D models.  On the other hand, some researchers\cite{Tian_2004} generated assessment metrics for textured 3D models by combining classical mesh and image metrics simply and objectively. However, whether either of these two assessment ways is correlated with subjective opinion has never been verified. Furthermore, to achieve the metrics for textured 3D meshes, which are highly correlated with subjective opinion, studies on the visual impact weights of the geometry and the texture are required.\\
In this context we propose a methodology of modeling mesh and texture metrics properly and efficiently, based on the subjective opinion. For this purpose, we firstly select several sets of textured 3D models applied with different types of distortions (geometric distortions: smoothing, quantization and simplification; textural distortions: JPEG and sub-sampling). Meanwhile, we design an efficient and reliable subjective experimental platform following the method from \cite{Farrell_2001}. During the experiment, we show the reference model to the observers at first, and then, all the distorted models randomly in pair. For each paired comparison, the observers are asked to select out the model closer to the reference one. In the end, the platform produces a subjective rank of all the distorted models given by each observer, which shows the ratings from the best visual quality to the worst one. After the analysis of the inter-observer agreement, we obtain some information showing human visual sensitivity on textured 3D mesh, and, propose a novel geometric visual quality metric according to our observations of experimental results. Then we select several recent perceptual geometric and textural (image) metrics for measuring the mesh and texture visual qualities respectively, and utilize the subjective opinion dataset to model these two kinds of metrics into one. Finally, we perform different quantitative validations and comparisons towards all the modeled metrics and previous work.\\
The rest of this paper is organized as follows. Section 2 provides a review of the related work about visual quality assessments on image, geometry and textured 3D mesh.  Section 3 describes our subjective experiment and the novel geometric visual quality metric according to our observations.
Section 4 demonstrates the evaluations of Image and mesh metrics, the methodology of modeling these two  metrics optimally, validations, and studies on the influence of the rendering. Finally, concluding remarks and perspectives are given in Section 5.