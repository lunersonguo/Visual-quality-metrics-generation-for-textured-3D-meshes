\section{Introduction}
Texture mapped 3D graphics are now commonplace in many fields of industry including digital entertainment, cultural heritage and architecture. They consist of geometric surfaces, on which are mapped several texture images that serve to make the rendering more realistic. Common texture maps include diffuse map, normal map and specular map. After their creation (by a designer or from a scanning/reconstruction process), these textured 3D assets maybe subject to diverse processing operations including simplification, compression, filtering, watermarking and so on. For instance, with the goal of accelerating the transmission for remote Web 3D visualization (e.g. for a virtual museum application), the geometry may be simplified and quantized, and the texture maps may be subject to JPEG compression. Similar geometry and texture degradations may also occur when these assets have to be adapted for lightweight mobile devices; in that case, textures may have to be sub-sampled or compressed using some GPU-friendy random-access methods (e.g. \cite{Strom2005}). These geometry and texture content corruptions may severely impact the visual quality of the 3D asset. Therefore, there is a need for efficient perceptual metrics able to evaluate the visual impact of these textured model artifacts on the visual quality of the rendered image.\\
Many visual quality metrics have been introduced in the field of computer graphics. However, most of them apply on images created after the rendering step. They mostly focus on detecting artifacts coming from global illumination approximation or tone mapping \cite{Aydin2010,Herzog2012,Yeganeh2013,Cadik2013}. On the contrary, another class of method focuses on evaluating the artifacts introduced on the 3D assets themselves, however most of them consider only geometric distortions \cite{Lavoue2011,Vasa2012,Wang2012}. Little work has been done to evaluate the visual impact of both geometry and texture distortions, on the appearance of the rendered image. Studying the involved complex perceptual interactions would require a ground-truth of subjective opinions on some models with such degradations. To our knowledge, only \citet{Pan2005} conducted such subjective study, however they considered only geometry and texture sub-sampling distortions. In this paper, we present a large-scale subjective experiment for this purpose, based on a paired comparison protocol. As in \cite{Pan2005}, we restrict the texture information to the diffuse maps. The dataset contains more than 232 videos of animated 3D models created from 5 reference objects, 5 types of distortions, 2 rendering parameters and the experiment involved more than 100 people. After an analysis of the influence of lighting on the perception of geometry and texture artifacts, we then use this subjective ground-truth to evaluate the performance of a large set of state-of-the-art metrics (dedicated to image, video and 3D models). We finally propose a new metric based on an optimal combination of geometric and image measures.\\ 



% In the research of the geometric study, researchers \cite{Lavou2011}\cite{V_a_2012}\cite{Wang_2011} proposed many kinds of mesh visual quality (MVQ) metrics recently. Previous studies \cite{Corsini_2013}\cite{Guo_2015} show that the results from these metrics are highly correlated with the human opinion, compared to classical metrics, such as Hausdorff distance and root mean square error. However, for textured 3D models, the geometric properties may no longer have the dominant visual impacts since they can be compensated by their textures.  On the aspect of the texture, comprehensive studies on 2D images have paved the way for evaluating the graphics attacks on texture.  Similar to the 3D metrics, image visual quality metrics, such as structural similarity (SSIM) \cite{Wang_2004}, multi-scale structural similarity (MS-SSIM) \cite{Zhou_Wang_2011}, etc., produce better results than classical metrics (e.g., mean square error (MSE) and peak signal to noise ratio (PSNR), etc.) in term of correlation with the human opinion. However, similarly, the situation becomes complex when the operations cause the distortions on geometry or/and texture. Thus, solely using single metric (either for the mesh or for the texture) is not reasonable. Towards the textured 3D model evaluation, some researches focused on the final rendering of the models. For instance, video quality assessment metrics (MSU VQMT, etc.)  \cite{Seshadrinathan_2010}, inherited from image quality assessment metrics \cite{Wang_2006}, evaluate the videos, containing textured 3D models.  On the other hand, some researchers \cite{Tian_2004} generated assessment metrics for textured 3D models by combining classical mesh and image metrics objectively. However, whether either of these two assessment ways is correlated with subjective opinion has never been verified. Furthermore, to achieve the metrics for textured 3D meshes, which are highly matched with subjective opinion, studies on the visual impact weights of the geometry and the texture are required.\\
The rest of this paper is organized as follows. Section 2 provides a review of the related work, while section 3 describes our subjective experiment and its results. Section 4 present a comprehensive evaluation of state-of-the-art image and mesh metrics with respect to our subjective ground-truth, and details our proposed perceptual measure and its validation. Finally, concluding remarks and perspectives are given in Section 5.