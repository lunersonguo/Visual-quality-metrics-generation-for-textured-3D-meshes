\section{Relatd work}
There exist various visual quality metrics on image, mesh and textured 3D mesh, and we summarize a brief survey on these metrics. We firstly provide some visual perception based image metrics. Then several visual geometric metrics, enlightened by these image metrics, are presented. Readers can also refer to \cite{Corsini_2013} for the details of these geometric metrics.  Meanwhile, we also supply existing metrics on textured 3D models. 
\subsection{Image quality metrics}
Most existing 2D image metrics are known as full-reference, meaning that a complete reference image is compared with a processed one. The most wildly used and simplest metric is the mean squared error (MSE), stem from PSNR. However, it is not well correlated with subjective visual opinion \cite{Snyder_1985} \cite{Teo}. Hence, early researchers \cite{Mannos_1974} made extensive efforts on the development of bottom-up visual quality measurement, simulating the functionality of human visual system (HVS) components. However, the HVS is very complex and non-linear, and thus, it makes the measurements relied on very strong assumptions and generalizations \cite{Baro_1995}\cite{Xing_2002}\cite{Ramos_2001}. In contrast, Pioneers ,such like Wang et al. \cite{Wang_2004}, proposed  top-down visual quality metrics, extracting structural information (luminance, contrast and structure) from the scenes of compared images. Such as structural similarity (SSIM) \cite{Wang_2004} and multi-scale structural similarity (MS-SSIM) \cite{Zhou_Wang_2011}, they make local differences of the structural attributes in different scales, and then pool the variances into one global score. These metrics can outperform the bottom-up ones.
\subsection{3D mesh visual quality metrics}

