\section{Toward an optimal metric for textured mesh quality assessment}
\subsection{Novel geometric metric}
Through observations of geometric distortions (smoothing, quantization and simplification), we found that the order of the distortions with the highest strengths from three types shows a regularity, which is that the model ranked with the worst visual quality always comes from either models processed by the strongest quantization method or those processed by the strongest simplification method, and the models with strongest smoothing distortions never appear in the end of ranks. For instance, among 20 distorted Hulk models, the distortion with the worst visual quality is the quantization with 7 bits(vote score: 0.91), while the distortion of 4 iterations smoothing has a fairly high subjective vote score(7.45) (see Figure 8).  Similar cases happen in other different models. This regularity implies some mechanisms of visual systems on sensing the variances of geometric surfaces, which means HVS is more sensitive to the variances from local areas (distortion caused by simplification or quantization) rather than to the global slight variances (distortion caused by smoothing). Meanwhile, previous study \cite{Guo_2015} shows that HVS is highly adapted for extracting curvature as surface structural information.\\
Based on the observations and previous studies, we propose a novel local distortion measurement by computing the amount of variances of curvature in local neighborhoods between two meshes (distorted mesh: $M_d$ and reference mesh $M_r$).  In details, at first we use fast matching method to establish a correspondence between $M_d$ and $M_r$, obtaining all the curvatures {$C_i$} from $M_d$ and their corresponding curvature {$\hat{C_i}$} on the surface $M_r$. For each vertex {$v_i$} from $M_d$, we obtain the standard deviation $\delta_i$ of local curvature differences in a small surface region $h$, in which there $K$ vertices and points are enclosed . For $M_d$ and $M_r$, $\delta_i$ is computed as follows:
\begin{equation}
\delta_i =\sqrt{\frac{1}{K}\sum_{j=1}^{K}{((\hat{C_j}-C_j)-E(\hat{C_j}-C_j))^2}}
\end{equation}
  
 $E(\hat{C_j}-C_j)$ is the mean value of curvatures differences in the region $h$.Refer to the work in \cite{Lavou__2011}, for each vertex we also measured the differences in different scales($h_i$) to capture the perceptually meaningful scales, and improve the efficiency and robustness of the metric. We took three scales $h_i\in\{2\epsilon, 3\epsilon, 4\epsilon\}$, where $\epsilon = 2.5\%$ of the max length of the bounding box of the model.  Then Multi-scale local distortion measure $\delta_M(v)$ was computed as: $\delta_M(v) = \frac{1}{n}\sum_{i=1}^n{\delta}_v^{h_i}$. $n$ is the number of scales (3 in our study). Finally, through extensive trials, we utilized root mean square (RMS) to pool all the local measurements together and obtained our global SDCD (Standard Deviation of Curvature Difference) value: $G\Delta M_d=(\frac{1}{|M_d|}\sum_{v\in M_d }{ \delta_M(v)^2})^\frac{1}{2}$. This metric emphasizes on variances of local roughness deviations, whereas the overall changes of a mesh (e.g., global shrink of a model) are not focally considered. The performance of this metric will be evaluated and validated in the following subsection.\\
\subsection{Image and mesh metric evaluation}
Since all the distorted models have mono-type of attacks(either geometric or textural attacks),the subjective vote scores can be also used as ground truths to evaluate the existing geometric and imaging measurements respectively, and consequently determine the most appropriate measurements for the optimal combinations. For all the distorted models, we firstly split them into two groups: one group only with geometric distortions and the other only with textural distortions. Then we select several commonly used perceptually geometric and image metrics to measure the correspondent models, which means the geometric metrics measure the models in geometric distortions group, while the image metrics measure the models in textural distortions group. For the measurements of geometric distortions, we choose one of the best performed perceptually-motivated metrics: MSDM2 \cite{Lavou__2011}, and our newly proposed metric: SDCD, whilst we select MSE, SSIM \cite{Wang_2004} and MSSSIM \cite{Wang} metrics for the measurements of textural distortions.  Since each model has a vote score from subjects, in each group, we consider all the vote score as ground truth to evaluate the performances of metrics.  


