\section{Toward an optimal metric for textured mesh quality assessment}
\subsection{Novel geometric metric}
Through observations of geometric distortions (smoothing, quantization and simplification), we found that the order of the distortions with the highest strengths from three types shows a regularity, which is that the model ranked with the worst visual quality always comes from either models processed by the strongest quantization method or those processed by the strongest simplification method, and the models with strongest smoothing distortions never appear in the end of ranks. For instance, among 20 distorted Hulk models, the distortion with the worst visual quality is the quantization with 7 bits(vote score: 0.91), while the distortion of 4 iterations smoothing has a fairly high subjective vote score(7.45) (see Figure 8).  Similar cases happen in other different models. This regularity implies some mechanisms of visual systems on sensing the variances of geometric surfaces, which means HVS is more sensitive to the variances from local areas (distortion caused by simplification or quantization) rather than to the global slight variances (distortion caused by smoothing). Meanwhile, previous study \cite{Guo_2015} shows that HVS is highly adapted for extracting curvature as surface structural information.\\
Based on the observations and previous studies, we propose a novel local distortion measurement by computing the amount of variances of curvature in local neighborhoods between two meshes (distorted mesh: $M_d$ and reference mesh $M_r$).  In details, at first we use fast matching method to establish a correspondence between $M_d$ and $M_r$, obtaining all the curvatures {$C_i$} from $M_d$ and their corresponding curvature {$\hat{C_i}$} on the surface $M_r$. For each vertex {$v_i$} from $M_d$, we compute the local distortion $\delta_i$ in a small surface region $h$, in which there $K$ vertices and points are enclosed. for each mesh,  as follows:
\begin{equation}
\delta_i =\sqrt{\frac{1}{K}\sum_{j=1}^{K}{((\hat{C_j}-C_j)-E(\hat{C_j}-C_j))^2}}
\end{equation}
  
 $E(\hat{C_j}-C_j)$ is the mean value of curvatures differences in the region $h$.
Refer to the work in \cite{Lavou__2011}, for each vertex we also measured the differences in different scales($h_i$) to capture the perceptually meaningful scales, and improve the efficiency and robustness of the metric. We took three scales $h_i\in\{2\epsilon, 3\epsilon, 4\epsilon\}$, where $\epsilon = 0.25\%$ of the max length of the bounding box of the model.  Then Multi-scale local distortion measure $\delta_M(v)$ was computed as:
\begin{equation}
 \delta_M(v) =  \frac{\sum_{i=1}^n{\delta}_v^{h_i}}{n}
 \end{equation}
where $n$ is the number of scales (3 in our study).
This metric measures roughness

