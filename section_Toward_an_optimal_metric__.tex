\section{Toward an optimal metric for mesh quality assessment}
Since we have obtained all subjective ranks for all the distorted models, appropriately using all these data is critical for evaluating all the existing and newly proposed metrics, and the study of combining geometric and textural metrics optimally. However, the subjective ranks cannot be directly used for these purposes, because ranking is usually regarded as a less accurate method of expressing how far the different distorted models may be to each other.  To solve this problem, we made a preference matrix \cite{Ledda_2005} to convert the all the subjective ranks into votes score. For example, suppose that we have $n$ distorted models (e.g., 5 models : $D_1$ to $D_5$) ready to be compared via our comparison protocol.   At each paired comparison, each subject’s “vote” for the more similar model to the reference is recorded as a value of 1. After all the pairs have been presented, we can record the results in a $n \times n$ matrix (Table 3).
\begin{table}[]
\centering
\caption{Example of preference matrix from one subject after presenting all the paired comparison of five distorted models }
\label{my-label}
\begin{tabular}{llllll}
            & $D_1$ & $D_2$ & $D_3$ & $D_4$ & $D_5$ \\
$D_1$ & 0           & 0           & 0           & 1           & 0           \\
$D_2$ & 1           & 0           & 0           & 1           & 0           \\
$D_3$ & 1           & 1           & 0           & 0           & 1           \\
$D_4$ & 0           & 0           & 1           & 0           & 1           \\
$D_5$ & 1           & 1           & 0           & 0           & 0          
\end{tabular}
\end{table}

\begin{table}[]
\centering
\caption{Example of preference matrix from one subject after presenting all the paired comparison of five distorted models}
\label{my-label}
\begin{tabular}{lllllll}
            & $D_1$ & $D_2$ & $D_3$ & $D_4$ & $D_5$ & Score \\
$D_1$ & 0           & 0           & 0           & 1           & 0           & 1     \\
$D_2$ & 1           & 0           & 0           & 1           & 0           & 2     \\
$D_3$ & 1           & 1           & 0           & 0           & 1           & 3     \\
$D_4$ & 0           & 0           & 1           & 0           & 1           & 2     \\
$D_5$ & 1           & 1           & 0           & 0           & 0           & 2    
\end{tabular}
\end{table}
In Table 1, we give an example that the value 1 of the cell in column $D_4$ and row $D_2$ means that the subject thinks the model with distortion $D_2$ is closer to the reference than the model with distortion $D_4$, and it is quite a reverse that the cell in column $D_2$ and row $D_4$ should have a value 0, meaning more different from the reference. From Table 1, we can calculate the overall preference score for each distorted model by summing up the values ($p_i$) in each row. The overall score ($VS$) per distortion per subject is computed as: