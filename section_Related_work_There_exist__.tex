\section{Related work}
There exist various visual quality metrics on image, mesh and textured 3D mesh, and we summarize a brief survey on these metrics. We firstly provide some visual perception based image metrics. Then several visual geometric metrics, enlightened by these image metrics, are presented. Readers can also refer to \cite{Corsini_2013} for the details of these geometric metrics.  Meanwhile, we also supply existing metrics on textured 3D models. 
\subsection{Image quality metrics}
Most existing 2D image metrics are known as full-reference, meaning that a complete reference image is compared with a processed one. The most widely used and simplest metric is the mean squared error (MSE), stem from PSNR. However, it is not well correlated with subjective visual opinion \cite{Snyder_1985} \cite{Teo}. Hence, early researchers \cite{Mannos_1974} made extensive efforts on the development of bottom-up visual quality measurement, simulating the functionality of human visual system (HVS) components. However, the HVS is very complex and non-linear, and thus, it makes the measurements relied on very strong assumptions and generalizations \cite{Baro_1995}\cite{Xing_2002}\cite{Ramos_2001}. In contrast, Pioneers, such like Wang et al. \cite{Wang_2004}, proposed  top-down visual quality metrics, extracting structural information (luminance, contrast and structure) from the scenes of compared images. Such as structural similarity (SSIM) \cite{Wang_2004} and multi-scale structural similarity (MS-SSIM) \cite{Zhou_Wang_2011}, they make local differences of the structural attributes in different scales, and then pool the variances into one global score. These metrics can outperform the bottom-up ones.
\subsection{3D mesh visual quality metrics}
Inspired by the visual image quality metrics, for the full-reference geometric metrics, various perceptually-motivated measurements were designed by scientific community.  Karni and Gotsman \cite{Karni_2000} are the pioneers who firstly tried to improve the performance of geometric distance measurements by merging the Root Mean Square (RMS) distance between corresponding vertices and their Laplacien coordinates. Recently, researchers extract some geometric attributes (e.g., curvature \cite{Lavou__2011}, dihedral angles \cite{V_a_2012}, Saliency \cite{Lee_2005}, etc.) as local computational operators, and calculate their differences at vertex or edge level. These metrics outperform the previous ones, moreover, the survey from \cite{Guo_2015} shows that the curvature based metrics demonstrate excellent performance  in term of the correlation with local perceptual opinion, and thus, concludes that HVS is more sensitive to capture curvature from mesh surface than other attributes.
\subsection{Quality assessment for textured 3D mesh}
Due to the scarcity of textured 3D mesh visual quality metrics, most quality assessments focus on the videos or screen displays, containing textured 3D contents. In fact, they were extended from images visual quality assessments \cite{Seshadrinathan_2010}. For instant, some video assessments \cite{Zhou_Wang} apply SSIM or MS-SSIM index to video frame-by-frame on the luminance component of the video, and compute the average SSIM or MS-SSIM index over all the frames as the visual quality score. Then, Pinson et al. \cite{Pinson_2004} proposed a more perceptual correlated but complex assessment, video quality metric (VQM), which calculates the perceptual changes in spatial, temporal and chrominance properties from spatial-temporal sub-regions of video streams, and then pools them into a single quality score.  Fast quality measure (FQM) \cite{Tian_2004} is one of few existing quality metrics specific to textured 3D mesh, which measures the geometric and textural deviations of paired textured models, and then, objectively combines them into one single score. However, the metric is compose of two classical measurements (mean square error (MSE) for the texture and Hausdorff distance for the mesh), and no subjective opinion was considered on modeling the two measurements.
